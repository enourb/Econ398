\documentclass{article}
\usepackage[utf8]{inputenc}
\usepackage{bbm}
\usepackage{amsmath}
\usepackage{graphicx}

\title{Econ 398 Model}
\author{Ethan Nourbash}
\date{February 2022}

\begin{document}

\maketitle

\section{Set Up}
My model is inspired by Diamond 2016, with several key changes to provide insight on the impact of remote work on worker location choice: (1) The place of residence is determined separately from place of work; (2) the local good price is endogenous; (3) the amenity utility function has two factors, commute time and density of entertainment venues, that are both endogenous; and (4) the proportion of work done from home is integrated as an exogenous variable.\\
Assume that worker $i$ has a job in the city center where he is paid a wage $W_{i}$. He chooses to live in region $p$ that offers him the most desirable bundle of local good prices and amenities. Let this be represented by a Cobb-Douglas utility function of local goods $M$ with price $R_{p}$, national goods $O$ with a price $P$, and local amenities $A_{i,p}$. Let $s_i$ be the worker i's amenity utility function.
\begin{equation}
U(M, O, p) := M^\alpha O^{1-\alpha}e^{s_i(A_{i,p})}
\end{equation}
\begin{equation}
\textrm{s.t } R_{p}M + PO \leq W_{i}
\end{equation}
\\
Let rent, $R_{p}$, be a function of distance from city center $c_p$ such that $R_{p}'(c_p) < 0$ and $R_{p}''(c_p) > 0$. In particular, we will assume that the price of local goods takes this particular functional form.
\begin{equation} \label{rent}
R_{p} := \beta_{R_0} + \beta_{R_1} c_p + \beta_{R_2} c_p^2
\end{equation}
Where $\beta_{R_0} > 0$ is the intercept, $\beta_{R_1} < 0$, and $\beta_{R_2} > 0$. This equation says that rent is highest in the city center ($c_p = 0$) and all other rent in the region is determined by a decreasing convex function of how long it takes to get to the city.\\ \\
Let the amenity vector, $A_{i,p}$, contain two variables: mean commute time to work, $c_{i,p}$, and density of entertainment venues, $e_{p}$. Let the amenity utility function, $s_{i}$, have the following properties: $\frac{\partial}{\partial c}s_{i}<0$ and $\frac{\partial}{\partial e}s_{i}>0$. Since mean commute time is partially determined by amount of work done at home, $A_{i,p}$ will respond endogenously to the amount of WFH individual i experiences. To achieve this goal, let $s_i$ have the following form
\begin{equation}
s_i(A_{i,p}) := \beta_{s_0} c_{i,p} + \beta_{s_1} e_{p}
\end{equation}
where $\beta_{s_0}$ and $\beta_{s_1}$ are preference coefficients such that $\beta_{s_0} < 0$ and $\beta_{s_1} > 0$. The interpretation of these signs is that the further worker i has to commute, the less happy he is and the larger the amount of entertainment venues around him, the happier he is.\\ \\
Let $c_{i,p}$ be a function of distance from city center, with the following functional form
\begin{equation}
c_{i,p} := (1-r_i) c_p
\end{equation}
Where $c_p$ is the commute time to city center from place of residence, $c_p > 0$, and $r_i$ is the exogenous share of work done remotely, $0 < r_i < 1$.\\
The assumption behind this specified commute time function is that workers care about the mean commute time they experience rather that just the commute time of traveling to the office, $c_p$. For example, let worker i live 40 minutes from his office, he works from home $20\%$ of the time. His mean commute time would be 32 minutes. When deciding whether or not to relocate, he considers his current commute to be 32 minutes rather than 40.\\ \\
For further simplification, assume that $e_{p}$ is a function of distance from city center.\\
\begin{equation}
e_{p} := \beta_{e_0} + \beta_{e_1} c_p + \beta_{e_2} c_p^2
\end{equation}
Similar to the rent function, $\beta_{e_0} > 0$ is the intercept, $\beta_{e_1} < 0$, and $\beta_{e_2} > 0$. This function states that entertainment venues are most prevalent at the city center and decreases convexly with respect to the amount of time it takes to get to the city.

\section{Evaluation}
\subsection{Marginal Utility Equivalence}
The first order conditions of the household problem yield the following:
$$\frac{U_M}{R_{p}} = \frac{U_O}{P}$$
$$\frac{\alpha M^{\alpha - 1} O^{1-\alpha}e^{s_g(A_{i,p})}}{R_{p}} = \frac{ (1-\alpha)M^{\alpha}O^{-\alpha}e^{s_g(A_{i,p})}}{P}$$
$$\frac{\alpha O}{R_{p}} = \frac{(1-\alpha)M}{P}$$
$$O = \frac{(1 - \alpha)}{\alpha}\frac{R_{p}}{P}M$$
\subsection{Budget Constraint}
Substituting first order condition results for $O$ in the budget constraint yields
$$R_{p}M + P\frac{(1 - \alpha)}{\alpha}\frac{R_{p}}{P}M \leq W_{i}$$
$$(1 + \frac{(1 - \alpha)}{\alpha})R_{p}M \leq W_{i}$$
Assuming worker is using his entire wage to solve for $M$, we have
$$M = \frac{W_{i}}{(1 + \frac{(1 - \alpha)}{\alpha})R_{p}}$$
\subsection{Utility Maximization}
After combining the equations, log-transforming, and simplifying (See Appendix 9.5), we get the following results:
\begin{equation*}
    \begin{aligned}
    \max_{p} ln(U) & = (1-\alpha)ln(\frac{(1 - \alpha)}{\alpha}) - ln(1 + \frac{(1 - \alpha)}{\alpha}) + (-\alpha)ln(\beta_{R_0} + \beta_{R_1} c_p + \beta_{R_2} c_p^2)\\
    &  - (1-\alpha)ln(P) + ln(W_{i}) + \beta_{s_0} ((1-r_i) c_p) + \beta_{s_1} (\beta_{e_0} + \beta_{e_1} c_p + \beta_{e_2} c_p^2)
    \end{aligned}
\end{equation*}


\section{Utility Maximization Algebra}
If the worker is maximizing his utility function, he is maximizing the log-transformed utility function.
$$\max_{M, O,p} ln(U) = \alpha ln(M) + (1-\alpha)ln(O) + s_i(A_{i,p})$$
Substitute in first order condition.
$$\max_{M, p} ln(U) = \alpha ln(M) + (1-\alpha)ln(\frac{(1 - \alpha)}{\alpha}\frac{R_{p}}{P}M) + s_i(A_{i,p})$$
$$\max_{M, p} ln(U) = \alpha ln(M) + (1-\alpha) \left [ln(\frac{(1 - \alpha)}{\alpha}) + ln(\frac{R_{p}}{P}) + ln(M) \right ] + s_i(A_{i,p})$$
$$\max_{M, p} ln(U) = \alpha ln(M) + (1-\alpha)ln(\frac{(1 - \alpha)}{\alpha}) + (1-\alpha)ln(\frac{R_{p}}{P}) + (1-\alpha)ln(M) + s_i(A_{i,p})$$
$$\max_{M, p} ln(U) = (1-\alpha)ln(\frac{(1 - \alpha)}{\alpha}) + (1-\alpha)ln(\frac{R_{p}}{P}) + ln(M) + s_i(A_{i,p})$$
Substitute in budget constraint.
$$\max_{p} ln(U) = (1-\alpha)ln(\frac{(1 - \alpha)}{\alpha}) + (1-\alpha)ln(\frac{R_{p}}{P}) + ln \left (\frac{W_{i}}{(1 + \frac{(1 - \alpha)}{\alpha})R_{p}} \right ) + s_i(A_{i,p})$$
$$\max_{p} ln(U) = (1-\alpha)ln(\frac{(1 - \alpha)}{\alpha}) + (1-\alpha)ln(\frac{R_{p}}{P}) + ln(W_{i}) - ln((1 + \frac{(1 - \alpha)}{\alpha})R_{p}) + s_i(A_{i,p})$$
\begin{equation*}
    \begin{aligned}
    \max_{p} ln(U) & = (1-\alpha)ln(\frac{(1 - \alpha)}{\alpha}) + (1-\alpha)ln(R_{p}) \\ 
     & - (1-\alpha)ln(P) + ln(W_{i}) - ln(1 + \frac{(1 - \alpha)}{\alpha}) - ln(R_{p}) + s_i(A_{i,p})
    \end{aligned}
\end{equation*}
$$\max_{p} ln(U) = (1-\alpha)ln(\frac{(1 - \alpha)}{\alpha}) - ln(1 + \frac{(1 - \alpha)}{\alpha}) + (-\alpha)ln(R_{p}) - (1-\alpha)ln(P) + ln(W_{i}) + s_i(A_{i,p})$$
Substitute in amenity preference functional form.
\begin{equation*}
    \begin{aligned}
    \max_{p} ln(U) & = (1-\alpha)ln(\frac{(1 - \alpha)}{\alpha}) - ln(1 + \frac{(1 - \alpha)}{\alpha}) + (-\alpha)ln(R_{p}) \\
    & - (1-\alpha)ln(P) + ln(W_{i}) + \beta_{s_0} c_{i,p} + \beta_{s_1} e_{p}
    \end{aligned}
\end{equation*}
Substitute in $R$, $c$, and $e$'s functional form.
\begin{equation*}
    \begin{aligned}
    \max_{p} ln(U) & = (1-\alpha)ln(\frac{(1 - \alpha)}{\alpha}) - ln(1 + \frac{(1 - \alpha)}{\alpha}) + (-\alpha)ln(\beta_{R_0} + \beta_{R_1} c_p + \beta_{R_2} c_p^2)\\
    &  - (1-\alpha)ln(P) + ln(W_{i}) + \beta_{s_0} ((1-r_i)\beta_r c_p) + \beta_{s_1} (\beta_{e_0} + \beta_{e_1} c_p + \beta_{e_2} c_p^2)
    \end{aligned}
\end{equation*}
\end{document}
